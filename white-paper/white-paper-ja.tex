\documentclass[xelatex,ja=standard]{bxjsarticle}

%% Language and font encodings
\usepackage[english]{babel}
\usepackage[utf8x]{inputenc}
\usepackage[T1]{fontenc}
\usepackage{tikz}

%% Sets page size and margins
\usepackage{geometry}
\geometry{top=3cm,bottom=2cm,left=3cm,right=3cm}

%% enable multiple citing at once
\usepackage{cite}
%% enable url in citing
\usepackage{url}

%% Useful packages
\usepackage{amsmath}
\usepackage{graphicx}
\usepackage[colorinlistoftodos]{todonotes}
\usepackage[colorlinks=true, allcolors=blue]{hyperref}
\usepackage{titling}
\usepackage[export]{adjustbox}
\usepackage{fancyhdr}
\usepackage{hyperref}

%% cross-ref list
%% https://tex.stackexchange.com/questions/1230/reference-name-of-description-list-item-in-latex
\makeatletter
\let\orgdescriptionlabel\descriptionlabel
\renewcommand*{\descriptionlabel}[1]{%
  \let\orglabel\label
  \let\label\@gobble
  \phantomsection
  \edef\@currentlabel{#1}%
  %\edef\@currentlabelname{#1}%
  \let\label\orglabel
  \orgdescriptionlabel{#1}%
}
\makeatother

%% Add logo
% https://www.overleaf.com/15705485shkbqqyzyxqs#/59720786/
% enable fancy page style on every page (the first page is excluded somehow?)
\pagestyle{fancy}
% first, clear default header
% https://texblog.org/2007/11/07/headerfooter-in-latex-with-fancyhdr/
\fancyhead{}  
% do not remove rule line
% \renewcommand{\headrulewidth}{0pt}
% \renewcommand{\footrulewidth}{0pt}
\setlength\headheight{55.0pt}
\addtolength{\textheight}{-55.0pt}
% add logo aligned to the left
\lhead{\adjincludegraphics[height=55pt, trim={0 {.2\height} 0 {.2\height}},clip]{logo-05.png}}


%% Add image above the title
\pretitle{
  \begin{center}
  \LARGE
  %  trim={<left> <lower> <right> <upper>}
  % width=\textwidth,
  \vspace{-5cm}
  \adjincludegraphics[width=\textwidth,trim={0 {.20\height} 0 {.20\height}},clip]{introduction2.png}\\
}
\posttitle{\end{center}}

\title{Crypko ホワイトペーパー\\
  \large 敵対的生成ネットワーク(GAN)を使った、クリプトコレクティブル・ゲーム\\
  \rightline{\small ver 0.8.0}
}
\author{}
\date{}

\begin{document}
\maketitle

\renewcommand\abstractname{概 要}
\begin{abstract}

2018年、AI(人工知能)とブロックチェーンが注目を集めています。

しかし、一般の人々は、理解出来ないと感じていたり、片方しか知らないというケースがほとんどです。研究者やマスコミは美しい青写真を描いていますが、人々はAIやブロックチェーンが今どんな発展段階にあり、これらの技術で何ができるのか、どのように生活が変わろうとしているのかを分かっていません。

他方、これらの2つの技術は、ほぼ別々に発展してきました。両者の垣根を超えてその強みを組み合わせるような仕事はほとんど行われて来ませんでしたし、一般人は完全に蚊帳の外でした。

このギャップに対して、私たちは、敵対的生成ネットワーク(GAN)を使ったクリプトコレクティブル・ゲームであるCrypkoを提案します。これこそが、AIとブロックチェーンの長所の融合です。Crypkoでは、具体的には
\begin{itemize}
\item ブロックチェーンとスマートコントラクトにより、デジタル資産の価値と所有権を保証します。
\item GANは、鑑賞と収集に値するクオリティの、多様なアニメキャラクターのアバターを生成します。
\item 生成ネットワークの特性を利用することで、スマートコントラクトによる2つの画像の融合が可能になります。融合の結果は追跡可能かつ予測不能であるため、ゲームがより面白いものになります。
\end{itemize}

\end{abstract}

% no page style on first page
\thispagestyle{empty}

\newpage

\section{背景}

2017年11月28日、CryptoKitties\cite{cryptokitties} の登場により、新しいタイプのDAPPであるクリプトコレクティブル・ゲームにスポットライトが当たりました。多くの会社やチームがクリプトコレクティブル・ゲームを発表しました\cite{cryptomons,cryptocountries,cryptopets,cryptoarts,cryptolandmarks,cryptofighters,etheremon,etherwaifu}。この手のゲームの特徴として、キャラクター\footnote{訳者注:原文は”Cryptocollectible”。収集されたアセットのひとつひとつを指すもので、ゲームによっては絵画や領土だったりしますが、堅苦しいので以下「キャラクター」と訳します}はブロックチェーンによって所有権と唯一性が保証されています。また、2つのキャラクターから新たな1つのキャラクターを作ることができるゲームは特に「ブリーダブル(交配可能)」と呼ばれます。
技術は進歩しているものの、既存のクリプトコレクティブル・ゲームには今なお、改善の余地が多く残っています。
\begin{description}
\item [問題1\label{problem:1}] ひとつひとつのキャラクターは、独立したいくつかのパーツで構成されています。ブリーダブルなものに関しては、独立したパーツがそれぞれ遺伝していくため、ちぐはぐなキャラクターが生まれてしまいます。
\item [問題2\label{problem:2}] キャラクターの見た目に多様性がありません。たいていのゲームではたかだか数百パターンのパーツの組み合わせであり、プレイヤーはたやすく飽きてしまいます。
\item [問題3\label{problem:3}] 運営側が設定する「レア度」によって、キャラクターの価値が決まってしまいます。しかし本来、価値というのは、そのキャラ自身が持つかけがえの無さによって決まるものです。
\end{description} 

\subsection{Cryptokitties による、新たなDAPPの夜明け}

Cryptokitties は、以下のアイデアを打ち出し、新たな DAPPS の先駆けとなりました。私たちはその革新性に賛辞を送り、そのアイデアを Crypko に継承していきたいと思います。

\begin{enumerate}
\item 既存のブロックチェーンで DAPP をリリースするほうが、簡単だし有益だし実用的。そういう意味では、ブロックチェーンプロジェクトをやる上で、ICO が唯一のファンディング方法ではなくなっています。
\item いわゆる代替不可能トークン(NFT)をサポートする ERC-721 の規格に準拠することで、キャラクターは不変で、代替不可能なものになります。これが、唯一性と価値の裏付けになります。
\item ゲーミフィケーションを通し、ややこしい技術用語抜きに、誰でもブロックチェーンの便利さとスゴさを体験することが出来ます。
\item ブリーディグ手数料、トランザクション手数料および0世代キャラクターの販売により、持続可能な収益構造になっています。
\item 2体の NFT から新たな NFT を生むことができるシステムにより、ゲームの面白みが増しています。
\end{enumerate}

\subsection{Cryptokittiesに続く、新たなクリプトコレクティブル・ゲーム}

Cryptokittiesのリリース後、それを発展させてクリプトコレクティブル・ゲームをデザインしようという、いくつかの試みがなされました。多くはスマートコントラクトに基づく、NFT間のバトルを導入するものでした\cite{cryptofighters,fishbank,cryptomons,etheremon} 。良い着眼点ではありましたが、ターン制やリアルタイム制のバトルをやるためには、イーサリアムチェーンのスマートコントラクト実行速度がネックとなりました。

他のクリプトコレクティブル・ゲームでは、NFT キャラクターを猫から他の動物へ、さらには生物の枠を飛び越え、領土や、名所や、絵画へと発展させました。しかし、これらのゲームではNFTを新たに作り出すことが出来ないので、ブリーディングの楽しみがありませんでした。

\subsection{クリプトコレクティブル・ゲームの限界}
\subsubsection{連続性のない遺伝と、多様性のないキャラクター}

NFTはオークションに掛けることで、市場価値が生まれます。それに加えて、新たなNFTを生む唯一の方法であるブリーディングが、カスタマーを離さず、持続的な収益を生む重要な鍵です。

イーサリアムでクリプトコレクティブル・ゲームをデザインする際には、両親から連続的に形質を遺伝させるべきです。ブリーディングによって生まれる子NFTは、現実世界のように、両親から見た目の特徴を(多寡はあれど)受け継ぐべきです。

これに対する既存のアプローチは、ざっくりと2通りに分けることが出来ます。(1)イラストレーターに遺伝のしかたの見本を見せながら、労働力を投入して力技で全NFTの絵を描く。(2)遺伝形質をいくつかの独立したパーツに分解し、組み合わせる。前者のアプローチは、クオリティの高いNFTキャラクターイメージを作ることができますが、新キャラを出す際には新たに絵を描く必要があるため、ブリーディングでまったく新しいNFTを生み出すことが出来ません。

もっと現実的ながら、ややクオリティの落ちる方法が、2番目のアプローチです。この方法もすでにいくつかのゲームで採用されています。この方法では、両親の遺伝子と突然変異との組み合わせによって、子NFTの遺伝子が決定されます。相互に独立した遺伝形質を組み合わせることにより、労働集約的に新たな絵を描く作業からは解放されます\cite{cryptokitties,etherwaifu,cryptofighters}。しかしこれではブリーディングは「着せ替えゲーム」にしかなりません。

さらに、パーツの組み合わせが膨大であることは、ユーザーが退屈しないことを意味しません。なぜならNFTごとの違いは、パーツの組み合わせの違いに過ぎず、パーツそれ自体の違いではないからです。ですが、実際イラストレーターが描けるパーツの数には限りがあり、たいていのキャラクターは10個以下のパーツの組み合わせから成り、それぞれのパーツのバリエーションも10個前後に留まります。それゆえ、このNFTはユニークだよと言われてもピンと来ないのです。

この問題に対し、ユニークに描き下ろした「レアキャラ」を数量限定で発行するという手法が生まれました。しかし、これらのレアキャラは遺伝的連続性から外れているのみならず、クリプトキャラの枠組みをも否定するものでした。

\subsubsection{価値=作られたレア度=パーツのレア度の合計 なのか?}

クリプトコレクティブル・ゲームのオークションにおいては、開発側がレアだと決めたものがレアであり、そういったパーツを持っているかどうかが、NFTの価格に直結します。

しかし現実世界のオークションでは、みんなが貴重だと思う商品に価値が付きます。クリプトコレクティブル・ゲームでも同じロジックが当てはまるのが自然です。しかし、依然として下記のような問題があります。


\begin{itemize}
\item キャラクターの持つ遺伝形質が実際珍しいものかどうかに関わらず、運営が「レア度」を設定してしまうと、価格決定に影響を与えてしまい、ひいてはゲームの分権性を損ないます。
\item ユーザーは自分の美的センスで「カワイイ」と思ったNFTを買います。でもマーケットで、美しいNFTよりも、運営がレアだと設定したNFTに価値が付いたらどうなるでしょう? NFTの本来の価値である美しさは置いてけぼりにされ、取引の道具に成り下がってしまうでしょう。
\end{itemize}

\subsection{敵対的生成ネットワーク(GAN)}
イアン・グッドフェロー氏によって考2014年に考案された敵対的生成ネットワーク(GAN)\cite{goodfellow2014generative}は、ディープラーニングアルゴリズムの一種で、 ニューラルネットワークを訓練することで、かなりホンモノに近いイメージを生成することが出来ます。それに続く一連の研究により、GANモデルでの画像生成は、本物っぽさも解像度も急速にレベルアップし\cite{radford2015unsupervised,karras2017progressive}、様々なアプリケーションへの活用が期待されています。

\subsubsection{MakeGirlsMoe〜AIでアニメキャラを作る!}
Crypko以前にも、私達はGANによるアニメキャラアバターの生成に着目していました。2017年8月、私達は\href{http://make.girls.moe/#/}{MakeGirlsMoe}をリリースしました。これはカスタマイズ可能なアニメキャラ生成システムです。

MakeGirlsMoe(MGM)では、GANモデルを我々の持つクリーンなアニメデータセットで訓練し、多様で、ハイクオリティで、カスタマイズ可能なアニメ顔イメージの生成を習得させました。最初のモデルである ``Amarylli'' はブラウザ上で $128 \times 128$サイズの画像が生成できました。2017年12月には、3番目のモデルである``Camellia''で解像度を$256 \times 256$ サイズまで引き上げました。

MGMのウェブサイトでは日本、アメリカ、中国、ロシアをはじめ、世界中から訪れた人々を楽しませることが出来ました。2018年2月1日には累計1500万アクセス、最大で23万アクセス/日を記録しました。

MGMは学会からの評価も受けており、ウェブサイトを公開して、MGMの背後にあるGANの手法をarXiv\cite{jin2017towards} に公開しました。 12月には、\href{https://nips2017creativity.github.io/}{NIPS 2017で、創造とデザインのための機械学習に関するワークショップ}においてプレゼンを行いました。

\subsubsection{GANの実用化における障害}

素直にGANを使おうとすれば、画像の生成を通して収益を得ることになります。ここではこれを「ピュアなGAN利用」と呼びましょう。例えば、生成した画像を売るのはピュアなGAN利用ですが、GANをデザインの補助に使うのはピュアなGAN利用ではありません。

MGMはGANを使った無料の、くり返し使われるサービスを意図しており、収益化は目指していませんでした。もっと一般的に言うと、あらゆるピュアなGAN利用にとって、共通する問題がありました。生成された画像は、どんなに精巧なものであれ、交換価値を持つのが困難だったのです。ツールによってナイスな絵をいくらでも生成できるからこそ、その乱用によって、生成された画像に希少性がなくなってしまったのです。それゆえ、GANの優位性を保つには、実際の使用に制限を設ける必要がありました。

生成された画像に交換価値を与えるには、以下の2つのアプローチが必要です。

\begin{enumerate}
\item 手数料を取る等で、生成ツールの乱用を防ぐ。
\item ユーザーに生成した画像の所有権を認めることで、希少性を保証する。
\end{enumerate}

方法1は直感的に理解できるでしょう。しかし方法2も併用しなければ、ユーザーは、購入した画像がネット上で勝手に拡散されたらどうしようという心配から逃れられません。ですが、方法2は実現不可能だと考えられてきました。ERC-721の登場までは。

\newpage

\section{製品}
Crypkoは、アニメキャラのクリプトコレクティブル・カードであり、ユーザーはトレードしたり、融合させて新たなカードを生み出したりできます。

すべてのCrypkoは、コードといくつかの属性(総称して「表現」)を固有のデータとして持っており、これに従い、ユニークな$512 \times 512$サイズの顔イラストが生成されます。属性はCrypkoのいくつかの外見上の特徴を明確に規定します。同時にコードによって、その他の特徴や細部が調整されます。\footnote{訳者注:たとえば青髪・緑眼・ツインテール・赤面というように、特徴を言語化して表しているのが属性で、それ以外の細部を決めているのがコードです。}

Crypkoの属性は、見た目にある程度反映されます。多くのクリプトコレクティブル・ゲームとは違い、まったく同じ属性を持つ2枚のCrypkoでも、違う見た目になります。

ユーザーは2枚のカード(オリジン)を融合し、新しいカード(デリバティブ)を作ることが出来ます。オリジンは無くなりませんが、融合後にはクールダウン期間になります。融合をするたび、クールダウンの期間は長くなります。融合の結果は予測できませんが、デリバティブは2枚のオリジンとある程度似ているはずです。自分の手持ちのカードでなく、他のユーザーから融合のためにカードを借りることもできます。

オリジンを持たないCrypkoは、イテレーション0です。イテレーション0のCrypkoは5万枚存在し、15分ごとに、自動的にオークションにリリースされます。運営は、はじめに、すべてのイテレーション0Crypkoのコードと属性を公開します。

理論上は、融合によっていくらでも新しいCrypkoを生むことが可能ですが、Crypkoの増殖にも制限を設けています。イテレーションの数字が大きいCrypkoや融合した回数の多いCrypkoほど、クールダウンの時間が長くなります。イテレーション0のCrypkoが全てリリースされた後は、融合が新たなCrypkoを生む唯一の方法です。

名前の由来について一言添えておきますと、Crypkoとは”Crypto”と”子”(子供を意味する日本語)からなる造語です。

\section{優位性}
CrypkoはGANを使うことにより、\ref{problem:1}〜\ref{problem:3}を解決しています。それによって、クリプトコレクティブル・ゲームがまだ到達していない領域へ踏み込もうとしています。

\subsection{連続的な融合}
融合は連続性を持っており、要素ごとに見ると、親子の顔のパーツはほとんど一致しませんが、全体で見ると遺伝的に一貫性のあるイメージになっています。

Crypkoでは融合は生成モデルの潜在空間上のベクトル算術演算\cite{radford2015unsupervised}に基づいています。ジェネレーターは表現ベクトルをインプットとして受け取り、それに対応するイメージをアウトプットします。ですから、2枚のイメージの遺伝子を適切に「融合」させたならば、両者の特徴を共有するイメージが出力されます。

これにより、私たちは\ref{problem:1}と\ref{problem:2}を解決できます。私たちのジェネレータは、アニメキャラのデータセットでのトレーニングを終えると、ドローイングの原則をマスターし、それぞれのCrypkoの特徴を見分けることが出来るようになります。その結果、融合は単純なパーツの組み合わせではなく、オリジンの持つコードに基づき、両方の特徴を受け継ぐ新たなCrypkoを一から描き上げることが出来るのです。

融合により、このランダムなゲームを少しだけコントロールできるようになります。ユーザーがカワイイと思うCrypko同士を融合させれば、新しいCrypkoもおそらく、カワイイでしょう。

私たちは、直感的な融合の連続性と、派生の多様性が、ユーザーを夢中にさせると確信しています。

\subsection{価値 = 萌え}

遺伝的連続性はあるものの、ジェネレーターが生成する「コード」を使うことにより、まったく新しい値付けのルールがもたらされます。

\subsubsection{潜在表現による複雑さ}

Crypko に「属性」があるのは検索とフィルタリングを便利にするために過ぎず、属性が Crypko の価値を決めるわけではありません。したがって、運営が属性のレアさを利用して市場操作することもありません。実際には、潜在ベクトルの一部である「コード」が Crypko の容姿を決定し、美しさや魅力を決定づけています。ですからCrypkoにおいては、他より優れた「属性」なんてものは存在しません。例えば髪の色が黒じゃなくブロンドだから価値が高い、なんてことは無くて、美しさを生む「コード」があるだけなんです。この「容姿」を決定するメカニズムは、ディープラーニングのブラックボックスの中であり、現在の技術では、誰もそのコードの意味を解き明かすことはできません。

\subsubsection{コードのインテグリティ}

解読不可能であるため、コードは部分部分にバラしても意味を持ちません。コードは全体が揃ってはじめて、価値あるコードなのです。あるCrypkoが美しいと感じた時、美しいのはCrypkoそれ自体であって、コードの部分部分ではありません。同様に、Crypkoの価値はそれ自体の価値であって、パーツに価値があるわけではありません。だから私たちは、キャラクターとしてのCrypkoの価値を、不連続で不自然なラベル付け\footnote{訳者注:属性のこと}から切り離します。

価格決定においては、美しさに直結しない「属性」か、解読不可能な「コード」しか材料がありません。ですからCrypkoの値段はユーザーの主観で決まることになります。属性と価格を分離することで、データから価格を計算するのが不可能になります。この条件の下では、プレイヤーは「自分がグッと来るか」のみを信じて値決めをすることになります。

これこそが、私たちが期待したクリプトコレクティブル・ゲームのあり方です。すなわちプレイヤーは主観的にCrypkoの価格を決めて、それに見合うトレードと融合を行うのみです。

このあり方は、他のクリプトコレクティブル・ゲームにも良い影響をもたらします。つまり

\begin{enumerate}
	\item ゲームの敷居が下がります。現実のオークション会場か、典型的なクリプトコレクティブル・ゲームをイメージして下さい。アカウントを作り、いくらかのお金をデポジットしたとしても、まだ心理的な障害があります。売買の経験や値決めに対する理解が不足しているため、ちゃんと価格に見合った買い物ができるか、心配になってしまうのです。Crypcoでは、構成要素は包括的でないし、パラメータは生成ネットワークと融合のランダム要素を含むので、ベテランが初心者より明らかに有利になることはありません。
	\item 価格付けについて客観的な判断ができない状態にしてあるので、運営がマーケットを操作することは困難です。たいていのクリプトコレクティブル・ゲームにおいては、運営が、特定の遺伝子があまり現れないように調整すると、それに対応する属性がレアになり、価格が上がります。しかし、価格決定のルールが主観に委ねられていれば、みんなが自分の好みに従って行動し、その結果、多くのプレイヤーから愛されるCrypkoに価値がつき、\ref{problem:3}が解決にになります。

\end{enumerate}

\section{コンセプト実証}
Crypkoのオープンβ版は、5月上旬にスタートする予定です。そこで私たちのコンセプトを明らかにし、ユーザーからのフィードバックを受け取り、UXとシステムの安定性向上を図ります。UIは\href{http://crypko.ai}{ crypko.ai }で公開し、PCやスマホから、いくつかのCrypkoが見られるようにします。

\href{https://metamask.io/}{MetaMaskプラグイン}をCromeまたはFirefoxに入れて、Rinkebyテストネットに接続し、アカウントを作成して下さい。それが済めば\href{https://www.rinkeby.io/#faucet}{Rinkebyフォーセット}からETHを入手できるので、トレードや融合をはじめとした、Crypkoの機能を試してみて下さい。

\section{結び}

GANは、画像生成のための汎用的な技術です。Crypkoは、クリプトコレクティブル・ゲームの新作という存在には留まらないと考えます。Crypkoをきっかけに、キャラクターに価値が内在し、楽しいと同時にオトナなゲームが、もっと市場に出てきてくれればと思います。

一方で、ERC-721によって、デジタル資産に価値を内在させることができました。またCrypkoがイーサリアムを使っているので、ディープラーニングの開発者たちは今以上に、GANやVAEやLSTMといった生成モデルをブロックチェーンゲームに載せ、AIの創造的役割を示そうとするでしょう。私たちは、最も適したトークンが生き残ると思いますが、生成モデルの発展のためにも、それがいちばん良いでしょう。

Crypkoの、ハイクオリティな生成モデルと直感的な価値評価基準によって、クリプトコレクティブル・ゲームに夢中になる人を増やせたらと思います。Crypkoにどこまで反響があるかは分かりませんが、ブロックチェーンとAIの組み合わせは、実用的であり、意味があり、そして楽しいことを証明できる自信はあります。

私たちはブロックチェーンの可能性を信じていますし、クリプトコレクティブル・ゲームに関しても同様です。レアキャラに価値があるのは事実ですが、人工的なレアキャラというラベルを、キャラクターに貼るべきではありません。息をのむような美しさと、多様な萌えがあるからこそ、愛おしいんじゃありませんか。

\vspace{5mm}

\begin{flushright}
 くりぷこチーム
 
 珈琲ねむりや訳
\end{flushright}

\renewcommand\refname{参考文献}
\bibliographystyle{plain}
\bibliography{bibliography}

\end{document}

